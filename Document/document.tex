%%%%%%%%%%%%%%%%%%%%%%%%%%%%%%%%%%%%%%%%%
% Medium Length Graduate Curriculum Vitae
% LaTeX Template
% Version 1.1 (9/12/12)
%
% This template has been downloaded from:
% http://www.LaTeXTemplates.com
%
% Original author:
% Rensselaer Polytechnic Institute (http://www.rpi.edu/dept/arc/training/latex/resumes/)
%
% Important note:
% This template requires the res.cls file to be in the same directory as the
% .tex file. The res.cls file provides the resume style used for structuring the
% document.
%
%%%%%%%%%%%%%%%%%%%%%%%%%%%%%%%%%%%%%%%%%

%----------------------------------------------------------------------------------------
%	PACKAGES AND OTHER DOCUMENT CONFIGURATIONS
%----------------------------------------------------------------------------------------

\documentclass[margin]{res} % Use the res.cls style, the font size can be changed to 11pt or 12pt here

\usepackage{helvet} % Default font is the helvetica postscript font
%\usepackage{newcent} % To change the default font to the new century schoolbook postscript font uncomment this line and comment the one above
\usepackage{enumitem}% http://ctan.org/pkg/enumitem
\usepackage{hyperref}
\usepackage{array}
\usepackage{cite}
%\usepackage[none]{hyphenat}%%%%
\newcolumntype{L}[1]{>{\raggedright\let\newline\\\arraybackslash\hspace{0pt}}p{#1}}
\newcolumntype{C}[1]{>{\centering\let\newline\\\arraybackslash\hspace{0pt}}p{#1}}
\newcolumntype{R}[1]{>{\raggedleft\let\newline\\\arraybackslash\hspace{0pt}}p{#1}}
\setlength{\textwidth}{5.1in} % Text width of the document

\begin{document}

    %----------------------------------------------------------------------------------------
    % NAME AND ADDRESS SECTION
    %----------------------------------------------------------------------------------------

    \moveleft.5\hoffset\centerline{\large\bf Wenhao Wu} % Your name at the top
 
    \moveleft\hoffset\vbox{\hrule width\resumewidth height 1pt}\smallskip %
    % Horizontal line after name; adjust line thickness by changing the '1pt'
 
    \moveleft.5\hoffset\centerline{2064 Kemper Hall, University of California,
    Davis} % Your address
    \moveleft.5\hoffset\centerline{Davis, CA, 95616}
    \moveleft.5\hoffset\centerline{wnhwu@ucdavis.edu}
    \moveleft.5\hoffset\centerline{(530) 601-3821}

    %----------------------------------------------------------------------------------------

    \begin{resume}

        %----------------------------------------------------------------------------------------
        % OBJECTIVE SECTION
        %----------------------------------------------------------------------------------------
     
        \section{OBJECTIVE}  
        A postdoc position in signal processing, wireless communication,
        data science and related fields.
    
        %----------------------------------------------------------------------------------------
        % EDUCATION SECTION
        %----------------------------------------------------------------------------------------
    
        \section{EDUCATION}
    
        {\sl \bf Ph.D. Candidate in Electrical and Computer Engineering} \hfill
        2012 - 2017 (expected) \\
        University of California, Davis, Davis, CA \\
        Advisor: Prof. Zhi Ding \\
        Cumulative GPA: 3.96
        
        {\sl \bf M.S. in Electrical and Computer Engineering}  \hfill 2012 -
        2014\\
        University of California, Davis, Davis, CA
        
        {\sl \bf B.S. in Electrical Engineering}  \hfill 2008 - 2012\\
        Tsinghua University, Beijing, China
       
        %----------------------------------------------------------------------------------------
        % TECHNICAL SKILLS SECTION
        %----------------------------------------------------------------------------------------
        \section{TECHNICAL \\ SKILLS}
        
        \begin{itemize}[leftmargin=*]
            \item Wireless Communication, Signal Processing, Statistical
            Learning.
            \item Java, Python, C/C++, Matlab/GNU Octave, R.
            \item Experience with HTML, XML, JavaScript, D3, CSS and SQL.
        \end{itemize} 
        
        %----------------------------------------------------------------------------------------
        % RESEARCH INTERESTS
        %---------------------------------------------------------------------------------------- 
        \section{RESEARCH INTERESTS}
        {\sl\bf Trans-Layer Design of RObust Header Compression:} (with
        Prof. Zhi Ding)\\
        Designing adaptive ROHC compressor exploiting various imperfect and
        delayed trans-layer information to take the optimal actions (e.g.
        compression level, feedback request,etc.) based on a POMDP framework
        \cite{wu2017efficient, wu2016efficient}.
        
        
        {\sl \bf Modulation Diversity for HARQ:} (with Prof. H. Mittelmann, 
        Prof. Zhi Ding)\\
        Modulation diversity design for various wireless channels with
        HARQ by solving Quadratic Assignment Problems (QAPs) to reduce bit error
        rate (BER) \cite{wu2016modulation2,
        wu2016modulation1, wu2016statistical}.
        
        {\sl \bf Cooperative multi-cell MIMO downlink precoding with
        finite-alphabet inputs:} (with Prof. Zhi Ding, Prof. Chengshan Xiao)\\
        Optimal linear precoder design for multi-cell MIMO downlink channel
        and finite-alphabet inputs with distributed implementation.
        \cite{wu2015cooperative, wu2014cooperative}.
        
        %----------------------------------------------------------------------------------------
        % PROFESSIONAL EXPERIENCE SECTION
        %----------------------------------------------------------------------------------------
     
        \section{EXPERIENCE}
        
        {\sl\bf Core Engineer Intern} \hfill 07/2014 - 09/2014 \\
        Range Networks, San Francisco, CA. Supervisor: James Peroulas.\\
        Implemented and tested the GMSK modulator of openBTS. Complete the
        software implementation of LTE PRACH receiver.
        
        {\sl\bf Teaching
        Assistant} \hfill 01/2013 - 04/2013 \\
        EEC 180A Digital Systems, Dept. ECE, UC Davis. Supervisor: Prof. Bevan
        Baas.
        
         %----------------------------------------------------------------------------------------
        % PROJECTS SECTION
        %---------------------------------------------------------------------------------------- 
        \section{COURSE PROJECTS}
        \href{https://github.com/huragok/STA208/tree/master/project}{{\bf NYC
        Taxi Data Pickup Prediction} (STA 208)} \hfill 06/2016 \\
        Predicting hourly taxi pickup per NTA in NYC using TLC trip record
        and NOAA weather dataset.
        
        \href{https://github.com/huragok/MAT280/tree/master/project}{{\bf
        Inferring the Night Life Hotspot from Taxi Trip Data} (MAT 280)} \hfill
        06/2016
        Clustering neighborhoods in NYC based on daily taxi
        pickup-dropoff patterns.
        
        \href{https://github.com/huragok/IDA}{\bf A Solution Manual for: The
        Elements of Statistical Learning } \hfill 01/2016
        
        \href{https://github.com/huragok/MAT228C}{{\bf An Introduction to
        Algebraic Multigrid} (MAT 228C)} \hfill 06/2013
        
        \href{https://github.com/huragok/ECS257}{{\bf WLAN RSSI-Based Indoor
        Localization And Tracking} (ECS 257)} \hfill 03/2013 \\
        Implementation and demonstration of various localization And
        tracking with real-world RSSI data and floor plan.
        
        
        
        %\pagebreak 
        %----------------------------------------------------------------------------------------
        % HONORS & AWARDS SECTION
        %----------------------------------------------------------------------------------------
        \section{HONORS \& \\ AWARDS}
        {\bf Third-class Excellent Social Service Scholarship} \hfill 10/2011 \\
        Tsinghua University, Beijing, China
        
        {\bf Honor of Excellent Leader of the Student Association for S\&T}
        \hfill 04/2011 \\
        Tsinghua University, Beijing, China
        
        {\bf First-class Outstanding Scholarship (Samsung Scholarship)}
        \hfill 12/2009 \\
        Tsinghua University, Beijing, China
        
        {\bf Second-class Outstanding Freshman Scholarship}
        \hfill 12/2008 \\
        Tsinghua University, Beijing, China
        
        %----------------------------------------------------------------------------------------
        % WEBSITES SECTION
        %----------------------------------------------------------------------------------------
    
        \section{WEBSITES} 
        LinkedIn: \url{https://www.linkedin.com/in/wenhao-wu-0ab2494b}\\
        Github: \url{https://github.com/huragok}
        
        %----------------------------------------------------------------------------------------
        % PUBLICATIONS SECTION
        %---------------------------------------------------------------------------------------- 
        \section{SELECTED PUBLICATIONS}
        \bibliographystyle{./IEEEtran}
        \renewcommand{\section}[2]{}
        \begin{thebibliography}{}
          \providecommand{\url}[1]{#1}
          \csname url@samestyle\endcsname
          \providecommand{\newblock}{\relax}
          \providecommand{\bibinfo}[2]{#2}
          \providecommand{ }{\spaceskip=0pt\relax}
          \providecommand{\BIBentryALTinterwordstretchfactor}{4}
          \providecommand{\BIBentryALTinterwordspacing}{\spaceskip=\fontdimen2\font plus
          \BIBentryALTinterwordstretchfactor\fontdimen3\font minus
            \fontdimen4\font\relax}
          \providecommand{\BIBforeignlanguage}[2]{{%
          \expandafter\ifx\csname l@#1\endcsname\relax
          \typeout{** WARNING: IEEEtran.bst: No hyphenation pattern has been}%
          \typeout{** loaded for the language `#1'. Using the pattern for}%
          \typeout{** the default language instead.}%
          \else
          \language=\csname l@#1\endcsname
          \fi
          #2}}
          \providecommand{\BIBdecl}{\relax}
          \BIBdecl
          
          \itemsep=4pt
          \bibitem{wu2017efficient} W. Wu, Z. Ding, ``On Efficient Packet
          Switched Wireless Networking: A Markovian Approach to Trans-layer Design of
          Bidirectional ROHC '', in preparation.
          \bibitem{wu2016efficient} W. Wu, Z. Ding, ``On Efficient Packet Switched
          Wireless Networking: A Markovian Approach to Trans-layer Design and
          Optimization of ROHC'', \emph{IEEE Trans. Wireless Commun.}, submitted
          for publication.
          \bibitem{wu2016modulation2} W. Wu, H. Mittelmann and Z. Ding,
          ``Modulation Design for MIMO-CoMP HARQ'', \emph{IEEE Commun. Lett.}, to be published.
          \bibitem{wu2016modulation1} W. Wu, H. Mittelmann and Z. Ding, ``Modulation Design for
          Two-Way Amplify-and-Forward Relay HARQ,'' \emph{IEEE Wireless
          Commun. Lett.}, vol. 5, no. 3, pp. 244-247, June 2016.
          \bibitem{wu2016statistical} W. Wu, H. Mittelmann, and Z. Ding, ``Statistical analysis of a
          posteriori channel and noise distribution based on HARQ feedback,”
          preprint: arXiv:1601.04131v1, 2016.
          \bibitem{wu2015cooperative} W. Wu, K. Wang, W. Zeng, Z. Ding, and C. Xiao,
          ``Cooperative multi-cell MIMO downlink precoding with
          finite-alphabet inputs,'' \emph{IEEE Trans. Commun.}, vol. 63, no.
          3, pp. 766-779, Mar. 2015.
          \bibitem{wu2014cooperative} W. Wu, K. Wang, Z. Ding, and C. Xiao, ``Cooperative multi-cell
          MIMO downlink precoding for finite-alphabet inputs,'' in
          \emph{IEEE Proc. Int. Conf. Acoust., Speech, Signal Process.}, May
          2014, pp. 464-468.
        \end{thebibliography}
        
    \end{resume}
\end{document}
